%\documentclass
\documentclass{article}

\usepackage{titlesec}
\usepackage{graphicx}
\usepackage{subcaption}
\usepackage{wrapfig}
\usepackage{caption}


\usepackage{multirow}

\titleformat{\section}[block]{\filcenter\Large\bfseries}{}{1em}{}
\captionsetup[figure]{labelformat=empty}

\title{Resumen Fisiología I}
\author{Ivan Dario Gonzalez Collazos}
\date{Septiembre 4 del 2023}

\begin{document}

\maketitle

\section{Conceptos básicos}

\begin{itemize}
    \item \textbf{Sonido}: Alteración física de un medio (solido, liquido o gaseoso), en donde se produce un movimiento oscilatorio de forma ondularía en las partículas, propagándose de manera omnidireccional desde la fuente sonora.
    \item \textbf{Transducir}: Transformación de la energía.
    \item \textbf{Aire}: Medio de propagación. 
    \item \textbf{Parlantes}: Transductores de salida, donde la energía eléctrica se transforma en energía mecánica cuando llega a los imanes del parlante y empieza a oscilar en la caja, transformándose esa energía de la oscilación en energía acústica cuando sale del parlante al medio.
    \item \textbf{Micrófono}: Transductor de entrada, donde la energía acústica se transforma en energía mecánica, moviendo los imanes y las membranas dentro del micrófono, transformando los movimientos de esos componentes en energía eléctrica.
    \item \textbf{Acústica}: Es todos los fenómenos que suceden en el medio de propagación del sonido, que en general es el su entorno natural, el aire.
    \item \textbf{Audio}: Representación en energía eléctrica del sonido, la cual puede ser manipulada. Una pista de audio es la representación de la energía eléctrica en el tiempo.
\end{itemize}

\textbf{\textit{Propiedades de la materia}}

\begin{itemize}
    \item \textbf{Masa}: es el peso representado en gramos ($g$) de un cuerpo.
    \item \textbf{Volumen}: Son las medidas (ancho, alto, largo) representadas en metros ($m$) de un cuerpo.
    \item \textbf{Elasticidad}: Es la propiedad de un cuerpo para volver a su forma original después de cambiar su forma. Se mide con el coeficiente de elasticidad ($k$).
\end{itemize}

\textbf{Movimiento oscilatorio}: Movimiento cíclico de un cuerpo alrededor de un eje.\\

\textbf{Onda}: Propagación de una alteración del medio en alguna propiedad del espacio, transportando energía, pero sin transportar la materia, transmitiendo esta energía por los átomos y partículas cerca al origen de la energía.\\

\textbf{Onda mecánica}: Son ondas que se propagan por medio de las propiedades mecánicas, de densidad y presión del medio.\\

\textbf{\textit{Tipos de ondas}}

\begin{itemize}
    \item \textbf{Transversales:} La materia y la energía se mueven de manera perpendicular entre ellas.
    \item \textbf{Longitudinales:} La materia y la energía se mueven de manera paralela entre ellas. 
    \item \textbf{Oblicuas:} Son ondas que se mueven de manera rotativa en el espacio.
\end{itemize}

El oído responde de mejor manera a las ondas longitudinales.\\

\textbf{\textit{Dominios de la ingeniera de sonido}}

\begin{itemize}
    \item \textbf{Dominio del tiempo:} Como se comporta la energía en el tiempo.
    \item \textbf{Dominio de la frecuencia:}  Representación gráfica (espectro) de cómo se comporta la amplitud a través de la frecuencia.
\end{itemize}

\section{Formula general de la oscilación}

\[ x = Asen(\omega t + \phi) \]

Donde:

\begin{itemize}
    \item $x$ es la cantidad de energía representada en $dB$ (decibeles).
    \item $t$ es el tiempo, que es una variable independiente.
    \item $A$ es $X_{max}$, que es la cantidad de energía máxima.
    \item $\omega t + \phi$ representa a la fase.
    \item $\omega$ Es la frecuencia, que tambien puede ser intepretada como velocidad angular.
\end{itemize}

\textbf{Frecuencia ($F$)}: Numero de ciclos por segundo. Se mide en $Hz$ (Herzios).\\

\textbf{Periodo ($T$)}: Tiempo medido en segundos ($s$), que tarda una partícula en dar un ciclo. Su formula es:

\[ T = \frac{1}{F} \]

Donde $F$ es la frecuencia.\\

\textbf{Angulo de fase ($\phi$)}: Posición inicial de la onda, expresada en grados o radianes.\\

\textbf{Fase ($\omega t + \phi$)}: Posición de la onda en cualquier instante de tiempo, expresada en grados o radianes.\\

\textbf{\textit{Velocidad angular ($\omega$)}}\\

Es la medida de la velocidad de rotación de un objeto (medida en $\frac{Rad}{s}$), que es define como:

\begin{equation}
   \omega = 2\Pi F
\end{equation}

Si expresamos $F$ como:

\begin{equation}
    F = \frac{1}{T}
\end{equation}

Podemos reemplazar (2) en uno y quedaria así:

\begin{equation}
    \omega = \frac{2\Pi}{T}
\end{equation}
 
\textbf{Fuerzas recuperadoras:} Son fuerzas que buscan generar un equilibrio a un sistema físico.\\

Newton define la fuerza como:

\begin{equation}
    \vec{F} = ma
\end{equation}

Donde $m$ es la masa de un objeto, y $a$ es su aceleración.\\

Hooke define la fuerza como:

\begin{equation}
    \vec{F} = -xk
\end{equation}

Donde $x$ es la distancia recorrida por el objeto, y $k$ es el coeficiente de elasticidad.\\

Si igualamos (4) y (5), tendríamos:

\begin{equation}
    ma= -xk
\end{equation}

Y si despejamos $a$ obtenemos la aceleración definida como:

\begin{equation}
    a= -\frac{xk}{m}
\end{equation}

Ahora, se tiene que derivar la formula general de la oscilación, para poder obtener la velocidad ($v$):

\begin{equation}
    \frac{\partial x}{\partial t} = A\omega cos(\omega t + \phi) = v
\end{equation}

Y si derivamos la velocidad, obtenemos la aceleración ($a$) que se expresa como :

\begin{equation}
    \frac{\partial v}{\partial t} = -A\omega^2 sen(\omega t + \phi) = a
\end{equation}

Con esto podemos reemplazar $a$ en (7) utilizando a (9) para reemplazar a $a$:

\begin{equation}
    -A\omega^2 sen(\omega t + \phi) = -\frac{xk}{m}
\end{equation}

Eliminamos el - y tenemos:

\begin{equation}
    A\omega^2 sen(\omega t + \phi) = \frac{xk}{m}
\end{equation}

Como en (11) tenemos $Asen(\omega t + \phi)$, podemos definirla como $x$ y tendríamos:

\begin{equation}
    \omega^2 x = \frac{xk}{m}
\end{equation}

Eliminamos $x$ de (12) y obtenemos:

\begin{equation}
    \omega^2 = \frac{k}{m}
\end{equation}

Para eliminar la potencia cuadrada, sacamos la raiz cuadrada a ambos lados:

\begin{equation}
    \omega = \sqrt{\frac{k}{m}}
\end{equation}

Podemos utilizar a (1) para reemplazar $\omega$, obteniendo:

\begin{equation}
    2\Pi F = \sqrt{\frac{k}{m}}
\end{equation}

Y podemos expresar la frecuencia como:

\begin{equation}
    F = \frac{\sqrt{k}}{2\Pi \sqrt{m}}
\end{equation}

Tambien podemos reemplazar $F$ utilizando (2), quedando:

\begin{equation}
    \frac{1}{T} = \frac{\sqrt{k}}{2\Pi \sqrt{m}}
\end{equation}

Y despejamos $T$ para obtener el tiempo:

\begin{equation}
    T = \frac{2\Pi \sqrt{m}}{\sqrt{k}}
\end{equation}

De la ecuación (16), podemos definir relaciones de proporcionalidad, donde podemos interpretar que:

\begin{itemize}
    \item Más masa es igual a tener frecuencias bajas.
    \item Menos masa es igual a tener frecuencias más altas.
\end{itemize}

\end{document}